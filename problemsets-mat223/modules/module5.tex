Consider the following situation: you're designing a 3d video game, but your uses only have 2d screens.
Or, you have a 900 dimensional dataset, but you want to visualize it on a continuum (i.e., as a line). 
Each of these is an example
of finding the best approximation to a set of points given some restrictions. In general, this operation
is called a \emph{projection} and in the world of linear algebra, it has a very particular meaning.

\SavedDefinitionRender{Projection}

Let $\mathcal P_{xy}\subseteq\R^3$ be the $xy$-plane in $\R^3$ and let $\vec v=\mat{1\\2\\3}$. Intuitively, 
$\Proj_{\mathcal P_{xy}}\!\!\vec v$ is the ``shadow'' that  $\vec v$ would casts on ${\mathcal P_{xy}}$ if 
the sun were directly overhead.
In other words,
\[
	\Proj_{\mathcal P_{xy}}\vec v=\mat{1\\2\\0}.
\]

XXX Figure

Let $\ell_y\subseteq \R^3$ be the $y$-axis in $\R^3$. It's a little bit harder to visualize what $\Proj_{\ell_y}\vec v$
is, so let's appeal to some definitions.

By definition, every vector in $\ell_y$ takes the form $\vec u_t=\mat{0\\t\\0}$ for some $t\in \R$. The distance
between $\vec u_t$ and $\vec v$ is
\[
	\Norm{\vec u_t-\vec v}=\norm*{\mat{0\\t\\0}-\mat{1\\2\\3}} = \sqrt{1^2+(t-2)^2+3^2}.
\]
Since $(t-2)^2$ is always non-negative, the quantity $\sqrt{1^2+(t-2)^2+3^2}$ is minimized when $(t-2)^2=0$. That is,
when $t=2$. Thus, we see $\vec u_2$ is the closest vector in $\ell_y$ to $\vec v$ and so,
\[
	\Proj_{\ell_y}\vec v=\vec u_2=\mat{0\\2\\0}.
\]

Here, we appealed directly to the definition of projection to find the answer.

\begin{example}
	Let $\ell\subseteq \R^2$ be the line given in vector form by $\vec x=t\mat{1\\1}+\mat{3\\-2}$,
	and let $\vec v=\mat{-1\\-1}$. Use the definition of projection to find $\Proj_\ell \vec v$.

	By definition, the distance between $\vec v$ and vectors in $\ell$ is given by
	\[
	    \Norm{\vec u-\vec v}=\norm*{\mat{t+3\\t-2}-\mat{-1\\-1}} = \sqrt{(t+4)^2+(t-1)^2}=\sqrt{2t^2+6t+17}=\sqrt{2\left(t+\frac{3}{2}\right)^2+\frac{25}{2}}.
	\]
	Since $(t+\frac{3}{2})^2$ is always non-negative, the value $t=-\frac{3}{2}$ minimizes the distance between $\ell$ and $\vec v$
	\[
	    \Proj_\ell \vec v=-\tfrac{3}{2}\mat{1\\1}+\mat{3\\-2}=\mat{3/2\\-7/2}.
	\]
\end{example}

Every example of a projection so far shares a geometric property. In the case of lines and planes,
the vector from the projection	to the original point is a normal vector for the line or plane.

XXX Figure

Stated precisely, if $X$ is a line or plane and $\vec v\notin X$ is a vector, then $\vec v-\Proj_X\vec v$ is a
normal vector for $X$. Using this fact, we can find projections onto lines and planes without solving
non-linear equations.

\begin{example}
	Let $\ell\subseteq \R^2$ be the line given in vector form by $\vec x=t\mat{1\\1}+\mat{3\\-2}$,
	and let $\vec v=\mat{-1\\-1}$. Use the fact that $\vec v-\Proj_\ell\vec v$ is a normal vector to $\ell$
	to find Find $\Proj_\ell \vec v$.

	Since $\vec v-\Proj_\ell\vec v$ is a normal vector to $\ell$, we know $\vec v-\Proj_\ell\vec v$ is orthogonal to $\vec d=\mat{1\\1}$. 
	
	Thus, let $\Proj_\ell\vec v = \mat{x\\y}$ for some unknown $x,y\in\R$, we have
	\[
	    (\vec v-\Proj_\ell\vec v)\cdot\vec d=\mat{-1-x\\-1-y}\cdot \mat{1\\1}=-2-x-y=0,
	\]
	that is, 
	\begin{equation}
	\label{EQ1}
	    x+y=-2.
	\end{equation}
	Also, since $\Proj_\ell\vec v$ is a vector in $\ell$, we know $\Proj_\ell\vec v$ also satisfies
	\[
    	\mat{x\\y}=t\mat{1\\1}+\mat{3\\-2}=\mat{t+3\\t-2}.
	\]
	By substituting $t=x-3$ into $y=t-2$, we get
	\begin{equation}
	\label{EQ2}
	    x-y=5.
	\end{equation}
	Solving the two-equations system given by \eqref{EQ1} and \eqref{EQ2}, we find $\Proj_\ell\vec v = \mat{3/2\\-7/2}$. 
\end{example}

\Heading{Projections Onto Other Sets}
For projections onto lines and planes, we can use what we know about normal vectors to simplify our life.
The same is true when projecting onto other sets, but we must always keep the definition in mind.

\begin{example}
	Let $\mathcal T\subseteq \R^2$ be the filled in triangle with vertices $\mat{0\\0}$, 
	$\mat{1\\0}$, and $\mat{0\\2}$, and let
	\[
		\vec a=\mat{1/4\\1/4}\qquad \vec b=\mat{3\\1/2}\qquad \vec c=\mat{1\\1}.
	\]
	Find $\Proj_{\mathcal T}\vec a$, $\Proj_{\mathcal T}\vec b$, and  $\Proj_{\mathcal T}\vec c$.

	XXX Finish
	XXX Figure
\end{example}

\Heading{Subtleties of Projections}
You might be wondering, what is $\Proj_X\vec v$ if $\vec v$ is equidistant from \emph{two}
closest points in $X$. Or, what if $X$ is an \emph{open} set (for example, an open interval in $\R^1$).
Then there might not be a closest point in $X$ to $\vec v$. In both these cases, we say $\Proj_X\vec v$
is \emph{undefined}.

Formally, for a fixed set $X$, we consider $P(\vec v)=\Proj_X\vec v$ as a function that inputs and outputs
vectors. And, as a function, $P$ has a domain consisting of exactly the vectors $\vec v$ for which $P(\vec v)$
is defined. As it happens, if $X$ is a line or a plane, the domain of $P$ is all of $\R^n$, and in this text,
we will be sensible and only ask about projections restricted to an appropriate domain. 

\Heading{Vector Components}

We've seen before that dot products can be used to measure how much one
vector points in the direction of another. But, we can go further. Suppose 
$\vec v\neq \vec 0$ and $\vec u$ are vectors. We might want to \emph{decompose}
$\vec u$ into the sum of two vectors, one which is in the direction of $\vec v$
and the other which is orthogonal to $\vec v$. The tool to that does this is the vector component.

\SavedDefinitionRender{Component}

From the definition, it's obvious that
\[
	\vec u=\Vcomp_{\vec v}\vec u + (\vec u-\Vcomp_{\vec v}\vec u)
\]
is a decomposition of $\vec u$ into the sum of two vectors, one, $\Vcomp_{\vec v}\vec u$, is in the direction of $\vec v$,
and the other, $\vec u-\Vcomp_{\vec v}\vec u$, is orthogonal to $\vec v$.

\begin{example}
	Find the component of $\vec a=\mat{1\\2}$ in the direction of $\vec b=\mat{1\\1}$.

    Let $\Vcomp_{\vec b}\vec a=\mat{x\\y}$ for some unknown $x,y\in\R$. 
    
    By definition, $\vec a - \Vcomp_{\vec b}\vec a$ is orthogonal to $\vec b$. Thus we have
    \[
        (\vec a-\Vcomp_{\vec b}\vec a)\cdot\vec b=\mat{1-x\\2-y}\cdot\mat{1\\1}=3-x-y=0,
    \]
    that is,
    \begin{equation}
    \label{EQ3}
        x+y=3.
    \end{equation}
    Also from the definition, $\Vcomp_{\vec b}\vec a$ is a vector in the direction of $\vec b$. Thus, for some $k\in\R$,
    \[
         mat{x\\y}=\Vcomp_{\vec b}\vec a = k\vec b=\mat{k\\k}.
    \]
    Since $x=y=k$, we get
    \begin{equation}
    \label{EQ4}
        x-y=0.
    \end{equation}
    Solving the two-equation system given by \eqref{EQ3} and \eqref{EQ4}, we obtain
    \[
        \vec a-\Vcomp_{\vec b}\vec a=\mat{3/2\\3/2}.
    \]
\end{example}

Since we'll be computing vector components often, let's try to find a formula for $\Vcomp_{\vec v}\vec u$.

By definition $\Vcomp_{\vec v}\vec u$ is a vector in the direction of $\vec v$, so
\[
	\Vcomp_{\vec v}\vec u = k\vec v.
\]
Further, from the definition $\vec u-\Vcomp_{\vec v}\vec u$ is orthogonal to $\vec v$, and so
\[
	\vec v\cdot (\vec u-\Vcomp_{\vec v}\vec u) = \vec v\cdot (\vec u-k\vec v)=\vec v\cdot \vec u-k\vec v\cdot \vec v=0,
\]
Because $\vec v\neq \vec 0$, we know $\vec v\cdot \vec v\neq 0$. Therefore, we may rearrange and solve for $k$ to find
\[
	k=\frac{\vec v\cdot \vec u}{\vec v\cdot \vec v},
\]
which means
\[
	\Vcomp_{\vec v}\vec u = \left(\frac{\vec v\cdot \vec u}{\vec v\cdot \vec v}\right)\vec v.
\]

\Heading{The Relationship Between Vector Components and Projections}

Vector components and projections onto lines are closely related. So closely related that many textbooks
use the single word \emph{projection} to talk about both vector components and projections. Let's take a moment
to explore this relationship.

Let $\vec v=\mat{2\\4}$ and $\vec u=\mat{2\\-1}$ and let $\ell=\Span\Set{\vec v}$. Drawing a picture of
$\ell$, $\vec u$, and $\Proj_\ell\vec u$, we see that $\Proj_\ell\vec u$ satisfies all the properties
of $\Vcomp_{\vec v}\vec u$.

XXX Figure

Since $\ell=\Span\Set{\vec v}$ and $\Proj_\ell\vec u\in\ell$, we know that $\Proj_\ell\vec u$ is in the direction
of $\vec v$. Further, from geometric arguments, we know $\vec u-\Proj_\ell\vec u$ is a normal vector for $\ell$
and is therefore orthogonal to its direction vector, $\vec v$! What's more, we didn't
use anything in particular about $\vec u$ and $\vec v$ when making this argument (other than $\vec v\neq \vec 0$). This means,
we may establish a general formula. 

\begin{theorem}
	For vectors $\vec u$ and $\vec v\neq 0$, we have
	\[
		\Proj_{\Span\Set{\vec v}}\vec u=\Vcomp_{\vec v}\vec u.
	\]
\end{theorem}

This is great news because vector components are easy to compute using dot products while projections are usually hard to compute.

\begin{example}
	Compute the projection of $\vec a=\mat{3\\7}$ onto $\ell=\Span\Set*{\mat{1\\-4}}$.

	Let $\vec b = \mat{1\\-4}$. 
	
	By the theorem above, for $\vec a$ and $\vec b\neq 0$, we have
	\[
	    \Proj_\ell \vec a = \Vcomp_{\vec b} \vec a = \left(\frac{\vec d\cdot \vec a}{\vec b\cdot \vec b}\right)\vec b = -\tfrac{3-28}{1+16}\mat{1\\-4}=-\tfrac{25}{17}\mat{1\\-4}.
	\]
\end{example}

It's worth noting, however, that vector components are equal to projections \emph{only in the case when you're
projecting onto a span}. In general, projections and vector components are unrelated. 

\begin{example}
	Let $\vec a=\mat{3\\7}$ , $\vec b=\mat{1\\-4}$, and let $\ell$ be the line given in vector form by
	$\vec x=t\vec b+\vec a$. Show that $\Proj_\ell\vec a\neq \Vcomp_{\vec b}\vec a$.

	By definition, $\Proj_{\ell}\vec a$ is the closest point in $\ell$ to $\vec a$. Computing the distance between $\ell$ and $\vec a$, we have
	\[
	    \norm*{(t\vec b+\vec a)-\vec a}=\Abs{t}\norm*{\vec b}=\sqrt{17}\Abs{t}.
	\]
	Since $\Abs{t}$ is always non-negative, the value $t=0$ minimizes the value of the distance. 
	
	Thus we obtain
	\[
	    \Proj_{\ell}\vec b = \vec 0 + \vec a = \mat{3\\7}.
	\]
	But by definition, $\Vcomp_{\vec b}\vec a$ is the vector in the direction of $\vec b$ so that $\vec a - \Vcomp_{\vec b}\vec a$ is orthogonal to $\vec b$. Therefore, $\Vcomp_{\vec b}\vec a$ would be the same as the example above. Thus, we find 
	\[
	    \Proj_{\ell}\vec b=\mat{3\\7}\neq-\tfrac{25}{17}\mat{1\\-4}=\Vcomp_{\vec b}\vec a.
	\]
	
\end{example}
